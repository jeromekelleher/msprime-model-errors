\documentclass{article}
\usepackage[round]{natbib}

\usepackage{listings}

\lstset{language=Python}

% local definitions
\newcommand{\msprime}[0]{\texttt{msprime}}
\newcommand{\ms}[0]{\texttt{ms}}
\newcommand{\stdpopsim}[0]{\texttt{stdpopsim}}


\begin{document}

\title{Describing demographic models is hard}
\author{A permutation of (AR, DN, JK, SG)}
\maketitle

\abstract{
We describe two errors made in defining population genetic models using the
\msprime\ coalescent simulator.

}

\section*{Introduction}

The \msprime\ coalescent simulator~\citep{kelleher2016efficient,kelleher2020coalescent}
is now quite widely used. The large increases in efficiency over the classical
\ms\ program~\citep{hudson2002generating} make it feasible to simulate large
samples of whole chromosomes for the first time. Another distinct advantage
of \msprime\ is the Python API that is its primary interface, greatly
increasing the flexibility and efficiency over the standard approach of
text-based command line interfaces. In particular, programs like \ms\
required users to specify cryptic command line options to describe demographic
models [JK: maybe include an example ms command line?]. Particularly for
the large models we are using today, these are not comprehensible for humans.
The Python API for \msprime\ is a great improvement, allowing the user to
state models in a documented and reproducible manner. [JK: maybe show
the same model as described above in msprime notation?]

Describing these models of population history is still hard and error
prone, however. In this note we discuss how two poor design decisions
in \msprime's demography API lead to errors being made, which ended
up in the scientific record.

\section{A bad tutorial example}

To illustrate the demography API, \msprime\ included a description of the
three population Out-of-Africa model~\citep{gutenkunst2009inferring}
as part of its tutorial documentation. This example model, however,
was not correctly implemented.

[What's wrong with the model, and how does this differ from the original. Some
analysis.]

This model was subsequently copied
several times and used in papers [JK: do we want to estimate how
many times it was copied? A quick search on github suggests around 30
times].

Arguably, this error occured because of a poor API design choice.
The MassMigration event is confusing and poorly explained. In
\msprime\ 1.0 we're introducing a PopulationSplit event, which allows
such models to be described more declaritively.

\section{A missing parameter}

[Description of what went wrong in the AJHG paper~\citep{martin2017human}.
Some analysis of how this affects the conclusions of the paper.]

[How was the error found? I wonder if the reason that Alicia didn't
do any basic checks on the simulation output was because it was
so massive. Until recently we couldn't do things like compute Fst, etc
on the tree sequences. Still, good enough reason to run small simulations
first to make sure that you're doing something sensible.]

This error was clearly caused by poor API design. By forcing the user to
pass around three separate parameters to describe the demographic model,
it's inviting simple mistakes such as this. In \msprime\ 1.0 we introduce
a Demography class, which wraps these three parameters.
Thus, instread of writing
\begin{lstlisting}[frame=single]
dbg = msprime.DemographyDebugger(
    population_configurations=population_configurations,
    migration_matrix=migration_matrix,
    demographic_events=demographic_events)
dbg.print_history()
ts = msprime.simulate(
    population_configurations=population_configurations,
    migration_matrix=migration_matrix,
    demographic_events=demographic_events)
\end{lstlisting}
we would now write
\begin{lstlisting}[frame=single]
dbg = demography.debug()
dbg.print_history()
ts = msprime.simulate(demography=demography)
\end{lstlisting}

\section{Conclusions}

Stuff to mention in no particular order:
\begin{itemize}
\item Defining demographic models is hard, and we need all the help we can get.
   We really need \stdpopsim.
\item If you do have to implement your own models, make sure that it is
verified. Getting someone else to implement it is a good approach (following
stdpopsim's QC procedure.)
\item It's really important to do some basic sanity checking on your
simulations. If your simulations are really big, do some small sims first and
analysis on this to make that basic properties hold.
\item Openness is essential. We only know about these errors because of
open code and open source development processes. There must be many, many more
out there.
\item API design matters! A lot!!
\end{itemize}

\bibliographystyle{plainnat}
\bibliography{paper}

\end{document}
